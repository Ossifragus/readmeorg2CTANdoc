% Intended LaTeX compiler: pdflatex
\documentclass{ltxdoc}
\usepackage[utf8]{inputenc}
\usepackage[T1]{fontenc}
\usepackage{graphicx}
\usepackage{longtable}
\usepackage{wrapfig}
\usepackage{rotating}
\usepackage[normalem]{ulem}
\usepackage{amsmath}
\usepackage{amssymb}
\usepackage{capt-of}
\usepackage{hyperref}
\usepackage{minted}

\author{Haim Bar and HaiYing Wang \\haim.bar@uconn.edu, haiying.wang@uconn.edu}
\ProvidesPackage{runcode}[2022/07/18 runcode v1.4]
\date{\today}
\title{The \textbf{runcode} package}
\begin{document}

\maketitle
\begin{abstract}

 \texttt{runcode} is a \LaTeX{} package that executes programming source codes
(including all command line tools) from \LaTeX{}, and embeds the results in
the resulting pdf file. Many programming languages can be easily used
and any command-line executable can be invoked when preparing the pdf
file from a tex file. \texttt{runcode} is also available on
\href{https://ctan.org/pkg/runcode}{CTAN}.

It is recommended to use this package in the server mode together with
the \href{https://www.python.org/}{Python}
\href{https://pypi.org/project/talk2stat/}{talk2stat} package. Currently,
the server mode supports \href{https://julialang.org/}{Julia},
\href{https://www.mathworks.com/products/matlab.html}{MatLab},
\href{https://www.python.org/}{Python}, and
\href{https://www.r-project.org/}{R}. More languages will be added.

For more details and usage examples and troubleshooting, refer to the
package’s github repository, at \url{https://github.com/Ossifragus/runcode}.

\end{abstract}

\section{Installation}
\label{installation}
You can simply put the runcode.sty file in the \LaTeX{} project folder.

The server mode requires the
\href{https://pypi.org/project/talk2stat/}{talk2stat} package. To install
it from the command line, use:

\begin{verbatim}
pip3 install talk2stat
\end{verbatim}

\textbf{Note}: \texttt{runcode} requires to enable the \texttt{shell-escape} option when
compiling a \LaTeX{} document.

\section{Usage}
\label{usage}
\subsection{Load the package:}
\label{load-the-package}
\begin{minted}[]{latex}
\usepackage[options]{runcode}
\end{minted}

Available options are:

\begin{itemize}
\item \texttt{cache}: use cached results.

\item \texttt{julia}: start server for \href{https://julialang.org/}{Julia} (requires
\href{https://pypi.org/project/talk2stat/}{talk2stat}).

\item \texttt{matlab}: start server for
\href{https://www.mathworks.com/products/matlab.html}{MatLab} (requires
\href{https://pypi.org/project/talk2stat/}{talk2stat}).

\item \texttt{nominted}: use the \href{https://ctan.org/pkg/fvextra}{fvextra} package
instead of the \href{https://ctan.org/pkg/minted}{minted} package to show
code (this does not require the \href{https://pygments.org/}{pygments}
package, but it does not provide syntax highlights).

\item \texttt{nohup}: use the \texttt{nohup} command when starting a server. When using
the server-mode, some editors terminate all child processes after
\LaTeX{} compiling such as Emacs with Auctex. This option set the
variable notnohup to be false, and the server will not be terminated
by the parent process. \textbf{This option has to be declared before
declaring any language}, e.g., \texttt{[nohup, R]} works but \texttt{[R, nohup]}
does not work.

\item \texttt{python}: start server for \href{https://www.python.org/}{Python}
(requires \href{https://pypi.org/project/talk2stat/}{talk2stat}).

\item \texttt{run}: run source code.

\item \texttt{R}: start server for \href{https://www.r-project.org/}{R} (requires
\href{https://pypi.org/project/talk2stat/}{talk2stat}).

\item \texttt{stopserver}: stop the server(s) when the pdf compilation is done.
\end{itemize}

\textbf{Note}: If \href{https://ctan.org/pkg/minted}{minted} is used, the style of
the code block is controlled through the minted package,
\href{https://github.com/Ossifragus/runcode/blob/master/examples/MontyHall/MontyHall.tex\#L3-L4}{e.g.:}

\begin{minted}[]{latex}
\setminted[julia]{linenos, frame=single, bgcolor=bg, breaklines=true}
\setminted[R]{linenos, frame=single, bgcolor=lightgray, breaklines=true}
\end{minted}

The outputs from executing codes are displayed in
\href{https://ctan.org/pkg/tcolorbox?lang=en}{tcolorbox}, so the style can
be customized with \texttt{\textbackslash{}tcbset},
\href{https://github.com/Ossifragus/runcode/blob/master/examples/MontyHall/MontyHall.tex\#L5}{e.g.:}

\begin{minted}[]{latex}
\tcbset{breakable,colback=red!5!white,colframe=red!75!black}
\end{minted}

\subsection{Basic commands:}
\label{basic-commands}
\begin{itemize}
\item \texttt{\textbackslash{}runExtCode\{Arg1\}\{Arg2\}\{Arg3\}[Arg4]} runs an external code.

\begin{itemize}
\item \texttt{Arg1} is the executable program.
\item \texttt{Arg2} is the source file name.
\item \texttt{Arg3} is the output file name (with an empty value, the counter
\texttt{codeOutput} is used).
\item \texttt{Arg4} controls whether to run the code. \texttt{Arg4} is optional with
three possible values: if skipped or with empty value, the value of
the global Boolean variable \texttt{runcode} is used; if the value is set
to \texttt{run}, the code will be executed; if set to \texttt{cache} (or anything
else), use cached results (see more about the cache below).
\end{itemize}

\item \texttt{\textbackslash{}showCode\{Arg1\}\{Arg2\}[Arg3][Arg4]} shows the source code, using
\href{https://ctan.org/pkg/minted}{minted} for a pretty layout or
\href{https://ctan.org/pkg/fvextra}{fvextra} (if
\href{https://pygments.org/}{pygments} is not installed).

\begin{itemize}
\item \texttt{Arg1} is the programming language.
\item \texttt{Arg2} is the source file name.
\item \texttt{Arg3} is the first line to show (optional with a default value 1).
\item \texttt{Arg4} is the last line to show (optional with a default value of
the last line).
\end{itemize}

\item \texttt{\textbackslash{}includeOutput\{Arg1\}[Arg2]} is used to embed the output from executed
code.

\begin{itemize}
\item \texttt{Arg1} is the output file name, and it needs to have the same value
as that of \texttt{Arg3} in \texttt{\textbackslash{}runExtCode}. If an empty value is given to
\texttt{Arg1}, the counter \texttt{codeOutput} is used.
\item \texttt{Arg2} is optional and it controls the type of output with a default
value \texttt{vbox}
\begin{itemize}
\item \texttt{vbox} (or skipped) = verbatim in a box.
\item \texttt{tex} = pure latex.
\item \texttt{inline} = embed result in text.
\end{itemize}
\end{itemize}

\item \texttt{\textbackslash{}inln\{Arg1\}\{Arg2\}[Arg3]} is designed for simple calculations; it runs
one command (or a short batch) and displays the output within the
text.

\begin{itemize}
\item \texttt{Arg1} is the executable program or programming language.
\item \texttt{Arg2} is the source code.
\item \texttt{Arg3} is the output type.
\begin{itemize}
\item \texttt{inline} (or skipped or with empty value) = embed result in text.
\item \texttt{vbox} = verbatim in a box.
\end{itemize}
\end{itemize}
\end{itemize}

\subsection{Language specific shortcuts:}
\label{language-specific-shortcuts}
\href{https://julialang.org/}{Julia}

\begin{itemize}
\item \texttt{\textbackslash{}runJulia[Arg1]\{Arg2\}\{Arg3\}[Arg4]} runs an external
\href{https://julialang.org/}{Julia} code file.
\begin{itemize}
\item \texttt{Arg1} is optional and uses
\href{https://pypi.org/project/talk2stat/}{talk2stat}'s
\href{https://julialang.org/}{Julia} server by default.
\item \texttt{Arg2}, \texttt{Arg3}, and \texttt{Arg4} have the same effects as those of the
basic command \texttt{\textbackslash{}runExtCode}.
\end{itemize}
\item \texttt{\textbackslash{}inlnJulia[Arg1]\{Arg2\}[Arg3]} runs \href{https://julialang.org/}{Julia}
source code (\texttt{Arg2}) and displays the output in line.
\begin{itemize}
\item \texttt{Arg1} is optional and uses the \href{https://julialang.org/}{Julia}
server by default.
\item \texttt{Arg2} is the \href{https://julialang.org/}{Julia} source code to run.
If the \href{https://julialang.org/}{Julia} source code is wrapped
between "\texttt{```}" on both sides (as in the markdown grammar), then it
will be implemented directly; otherwise the code will be written to
a file on the disk and then be called.
\item \texttt{Arg3} has the same effect as that of the basic command \texttt{\textbackslash{}inln}.
\end{itemize}
\end{itemize}

\href{https://www.mathworks.com/products/matlab.html}{MatLab}

\begin{itemize}
\item \texttt{\textbackslash{}runMatLab[Arg1]\{Arg2\}\{Arg3\}[Arg4]} runs an external
\href{https://www.mathworks.com/products/matlab.html}{MatLab} code file.
\begin{itemize}
\item \texttt{Arg1} is optional and uses
\href{https://pypi.org/project/talk2stat/}{talk2stat}'s
\href{https://www.mathworks.com/products/matlab.html}{MatLab} server by
default.
\item \texttt{Arg2}, \texttt{Arg3}, and \texttt{Arg4} have the same effects as those of the
basic command \texttt{\textbackslash{}runExtCode}.
\end{itemize}
\item \texttt{\textbackslash{}inlnMatLab[Arg1]\{Arg2\}[Arg3]} runs
\href{https://www.mathworks.com/products/matlab.html}{MatLab} source code
(\texttt{Arg2}) and displays the output in line.
\begin{itemize}
\item \texttt{Arg1} is optional and uses the
\href{https://www.mathworks.com/products/matlab.html}{MatLab} server by
default.
\item \texttt{Arg2} is the
\href{https://www.mathworks.com/products/matlab.html}{MatLab} source
code to run. If the
\href{https://www.mathworks.com/products/matlab.html}{MatLab} source
code is wrapped between "```" on both sides (as in the markdown
grammar), then it will be implemented directly; otherwise the code
will be written to a file on the disk and then be called.
\item \texttt{Arg3} has the same effect as that of the basic command \texttt{\textbackslash{}inln}.
\end{itemize}
\end{itemize}

\href{https://www.r-project.org/}{R}

\begin{itemize}
\item \texttt{\textbackslash{}runR[Arg1]\{Arg2\}\{Arg3\}[Arg4]} runs an external
\href{https://www.r-project.org/}{R} code file.
\begin{itemize}
\item \texttt{Arg1} is optional and uses
\href{https://pypi.org/project/talk2stat/}{talk2stat}'s
\href{https://www.r-project.org/}{R} server by default.
\item \texttt{Arg2}, \texttt{Arg3}, and \texttt{Arg4} have the same effects as those of the
basic command \texttt{\textbackslash{}runExtCode}.
\end{itemize}
\item \texttt{\textbackslash{}inlnR[Arg1]\{Arg2\}[Arg3]} runs \href{https://www.r-project.org/}{R}
source code (\texttt{Arg2}) and displays the output in line.
\begin{itemize}
\item \texttt{Arg1} is optional and uses the \href{https://www.r-project.org/}{R}
server by default.
\item \texttt{Arg2} is the \href{https://www.r-project.org/}{R} source code to run.
If the \href{https://www.r-project.org/}{R} source code is wrapped
between "```" on both sides (as in the markdown grammar), then it
will be implemented directly; otherwise the code will be written to
a file on the disk and then be called.
\item \texttt{Arg3} has the same effect as that of the basic command \texttt{\textbackslash{}inln}.
\end{itemize}
\end{itemize}

\href{https://www.python.org/}{Python}

\begin{itemize}
\item \texttt{\textbackslash{}runPython[Arg1]\{Arg2\}\{Arg3\}[Arg4]} runs an external
\href{https://www.python.org/}{Python} code file.
\begin{itemize}
\item \texttt{Arg1} is optional and uses
\href{https://pypi.org/project/talk2stat/}{talk2stat}'s
\href{https://julialang.org/}{Julia} server by default.
\item \texttt{Arg2}, \texttt{Arg3}, and \texttt{Arg4} have the same effects as those of the
basic command \texttt{\textbackslash{}runExtCode}.
\end{itemize}
\item \texttt{\textbackslash{}inlnPython[Arg1]\{Arg2\}[Arg3]} runs
\href{https://www.python.org/}{Python} source code (\texttt{Arg2}) and displays
the output in line.
\begin{itemize}
\item \texttt{Arg1} is optional and uses the \href{https://www.python.org/}{Python}
server by default.
\item \texttt{Arg2} is the \href{https://julialang.org/}{Julia} source code to run.
If the \href{https://www.python.org/}{Python} source code is wrapped
between "```" on both sides (as in the markdown grammar), then it
will be implemented directly; otherwise the code will be written to
a file on the disk and then be called.
\item \texttt{Arg3} has the same effect as that of the basic command \texttt{\textbackslash{}inln}.
\end{itemize}
\item \texttt{\textbackslash{}runPythonBatch[Arg1][Arg2]} runs an external
\href{https://www.python.org/}{Python} code file in batch mode (without a
server running). Python (at least currently), unlike the other
languages we use, does not have an option to save and restore a
session, which means that once a Python session ends, the working
environement (variable, functions) is deleted. In order to allow a
batch-mode in Python, we implemented such capability. It requires the
\href{https://pypi.org/project/dill/}{dill} module, which has to be
installed via \texttt{pip3 install dill}.
\begin{itemize}
\item \texttt{Arg1} is the \href{https://www.python.org/}{Python} source file name,
\item \texttt{Arg2} is the output file name.
\end{itemize}
\end{itemize}


\section{Contributing}
\label{contributing}
We welcome your contributions to this package by opening issues on
GitHub and/or making a pull request. We also appreciate more example
documents written using \texttt{runcode}.


\textbf{Citing \texttt{runcode}:} /Haim Bar and HaiYing Wang (2021). Reproducible
Science with \LaTeX{},
[\url{https://jds-online.org/journal/JDS/article/103/info}] J. data sci. 2021;
19, no. 1, 111-125, DOI 10.6339/21-JDS998/
\end{document}